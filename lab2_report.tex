\documentclass[12pt,a4paper]{article}

\usepackage[utf8]{inputenc}
\usepackage[margin=1in,headheight=15pt]{geometry}
\usepackage{graphicx}
\usepackage{amsmath}
\usepackage{amssymb}
\usepackage{hyperref}
\usepackage{booktabs}
\usepackage{xcolor}
\usepackage{tcolorbox}
\usepackage{enumitem}
\usepackage{fancyhdr}
\usepackage{titlesec}

% Color definitions
\definecolor{primarycolor}{RGB}{0,102,204}
\definecolor{secondarycolor}{RGB}{51,51,51}
\definecolor{highlightcolor}{RGB}{255,235,59}
\definecolor{keycolor}{RGB}{0,150,136}
\definecolor{notecolor}{RGB}{255,152,0}

% Hyperlink colors
\hypersetup{
    colorlinks=true,
    linkcolor=primarycolor,
    citecolor=primarycolor,
    urlcolor=primarycolor
}

% Section formatting
\titleformat{\section}
  {\Large\bfseries\color{primarycolor}}{\thesection}{1em}{}[\titlerule]
\titleformat{\subsection}
  {\large\bfseries\color{secondarycolor}}{\thesubsection}{1em}{}

% Header and footer
\pagestyle{fancy}
\fancyhf{}
\fancyhead[L]{\textcolor{primarycolor}{Lab 2 - Car-Sharing System Model}}
\fancyhead[R]{\textcolor{secondarycolor}{\thepage}}
\renewcommand{\headrulewidth}{0.5pt}
\renewcommand{\headrule}{\hbox to\headwidth{\color{primarycolor}\leaders\hrule height \headrulewidth\hfill}}

% Custom boxes
\newtcolorbox{keybox}{
  colback=keycolor!10,
  colframe=keycolor,
  boxrule=1pt,
  arc=3pt,
  left=5pt,
  right=5pt,
  top=5pt,
  bottom=5pt
}

\newtcolorbox{notebox}{
  colback=notecolor!10,
  colframe=notecolor,
  boxrule=1pt,
  arc=3pt,
  left=5pt,
  right=5pt,
  top=5pt,
  bottom=5pt,
  title={\textbf{Important Note}}
}

\newtcolorbox{summarybox}{
  colback=primarycolor!5,
  colframe=primarycolor,
  boxrule=1.5pt,
  arc=3pt,
  left=8pt,
  right=8pt,
  top=8pt,
  bottom=8pt
}


\title{
    \vspace{-1cm}
    {\Huge\textbf{\textcolor{primarycolor}{Computer Aided Simulation Lab}}} \\[0.5cm]
    {\LARGE Simulation of a Car Sharing System} \\[0.3cm]
    {\Large Labs 3 and 4}
}

\author{Riccardo Bracciale \\ S338616}


\begin{document}

\maketitle
\thispagestyle{empty}

\begin{summarybox}
\textbf{Laboratory Objectives:}
\begin{itemize}[leftmargin=*]
    \item Identify key system parameters and performance indicators
    \item Design a balanced simulation model (accuracy vs. complexity)
    \item Propose realistic user mobility patterns
    \item Define appropriate data structures for simulation
    \item Document all modeling assumptions
\end{itemize}
\end{summarybox}

\vspace{1cm}

\section*{Purpose of the Laboratory}

The purpose of this laboratory is to develop a simulation model that captures the operational dynamics of a \textbf{car-sharing system}. 

In such a system, users appear at random locations within an urban area and use a smartphone application to locate and reserve a nearby available car. Once a reservation is successful, the user drives the car to the desired destination and leaves it either in a designated parking area, such as a recharging station, or in any permissible location within the operational zone.

In addition to user-driven trips, cars can be partially relocated across the city to better align supply with spatial demand imbalances. This relocation mechanism is crucial to maintaining acceptable availability levels throughout the service area.


The following sections identify the fundamental system parameters, define the main performance indicators, outline the conceptual model design, and describe the data structures required to support the simulation.

\newpage

\section{Model Design}

\begin{keybox}
\textbf{Key Design Principle:} The simulation model is based on \textbf{discrete-event simulation} principles, where the system evolves through a sequence of discrete events rather than continuous time steps.
\end{keybox}

\vspace{0.5cm}

The whole system is represented as a collection of \textbf{entities} (users, vehicles, charging stations) that interact over time through a series of \textbf{events} (user arrivals, vehicle reservations, trip completions, charging operations, and relocations).

\subsection{User Behavior Model}

Specifically, users are expected to sign up for the platform. Once registered, users will randomly make requests to rent a car, with random destinations inside the service area.

\textbf{Reservation Process:}
\begin{enumerate}[leftmargin=*, label=\textcolor{primarycolor}{\arabic*.}]
    \item User makes a car request at a random location
    \item System searches for available cars within a search radius
    \item User is assigned to the \textit{closest available car}
    \item If no cars are available, user retries (up to a maximum number of attempts)
    \item After exhausting attempts, user abandons the request
\end{enumerate}

\subsection{Trip and Energy Dynamics}

The time it takes to travel between the user's location and the destination depends on the \textcolor{keycolor}{\textbf{traffic conditions}} between those two points, which can change over time. During the trip, the car uses up energy based on the distance driven.


If a car's battery is below a certain level at the end of a rental, it will be taken to be charged by a special worker called a \textit{relocator}. This person will pick up the car and drive it to the nearest charging station.


Once the car is fully charged, it will be available for new rentals again.




\section{Assumptions about the Model}

\begin{notebox}
All modeling decisions involve trade-offs between realism and computational tractability. The following assumptions define the boundaries of our simulation model.
\end{notebox}

\vspace{0.5cm}

In order to create a manageable and effective simulation model for the car-sharing system, several key assumptions are made:

\subsection{Spatial and Network Assumptions}

\begin{itemize}[leftmargin=*, label=\textcolor{primarycolor}{$\bullet$}]
    \item \textbf{Urban Area:} The service area is represented as a bounded two-dimensional space (10km $\times$ 10km) where users and vehicles can be located.
    
    \item \textbf{Road Network:} A graph structure represents the road network, with:
    \begin{itemize}
        \item \textit{Nodes} as intersection locations
        \item \textit{Edges} as possible paths for vehicles
        \item \textit{Weights} representing traffic conditions or travel time
        \item Traffic conditions may vary over time (rush hours, etc.)
    \end{itemize}
\end{itemize}

\subsection{User Behavior Assumptions}

\begin{itemize}[leftmargin=*, label=\textcolor{primarycolor}{$\bullet$}]
    \item \textbf{Arrival Process:} User arrivals follow a \textcolor{keycolor}{Poisson process} with a time-varying average rate
    
    \item \textbf{Trip Pattern:} After signup, users request cars for trips to random destinations within the service area, following a Poisson request pattern
    
    \item \textbf{Assignment Logic:} Users are assigned to the \textit{nearest available car} within a search radius
    
    \item \textbf{Retry Mechanism:} If no car is available, users retry multiple times before abandoning the request
\end{itemize}

\subsection{Vehicle and Infrastructure Assumptions}

\begin{itemize}[leftmargin=*, label=\textcolor{primarycolor}{$\bullet$}]
    \item \textbf{Fleet Characteristics:}
    \begin{itemize}
        \item Fixed fleet size of electric vehicles
        \item Each vehicle has maximum battery capacity (60 kWh)
        \item Constant energy consumption rate per kilometer (0.15 kWh/km)
    \end{itemize}
    
    \item \textbf{Charging Infrastructure:}
    \begin{itemize}
        \item Fixed number of charging stations distributed across the city
        \item Each station has limited charging capacity
        \item Charging rate: 7.2 kW (Level 2 AC charging)
    \end{itemize}
    
    \item \textbf{Relocation:} Special workers (relocators) pick up low-battery vehicles and drive them to the nearest charging station
\end{itemize}



\section{Key Parameters and Performance Indicators}

The system is characterized by a set of parameters that define its operational configuration and user behavior. These parameters form the foundation upon which the simulation is built and determine the relationships between demand, supply, and service performance.

\subsection{System Parameters}

\begin{table}[h]
\centering
\begin{tabular}{@{}lp{10cm}@{}}
\toprule
\textcolor{primarycolor}{\textbf{Category}} & \textbf{Description} \\
\midrule
\textcolor{keycolor}{Fleet Size} & Total number of vehicles available in the system. Determines potential capacity and directly influences availability and utilization. \\[0.3cm]

\textcolor{keycolor}{User Arrival Rate} & Rate at which users appear (Poisson process with mean rate per unit time). Affects system load and congestion. \\[0.3cm]

\textcolor{keycolor}{Trip Distribution} & Spatial distribution of origins/destinations and trip distances. Affects vehicle circulation and energy consumption. \\[0.3cm]

\textcolor{keycolor}{Charging Infrastructure} & Number and location of charging stations, plus charging rate. Constrains vehicle availability due to downtime. \\[0.3cm]

\textcolor{keycolor}{Relocation Strategy} & Frequency and magnitude of vehicle relocations. Influences response to geographical demand imbalances. \\
\bottomrule
\end{tabular}
\caption{Key system parameters influencing performance}
\end{table}

\textbf{Detailed Parameter Specifications:}

\begin{itemize}[leftmargin=*, label=\textcolor{primarycolor}{$\circ$}]
    \item \textbf{Fleet size:} Total number of vehicles available in the system (baseline: 20 cars)
    
    \item \textbf{User arrival rate:} Poisson process with time-varying rate (baseline: 0.15 users/hour)
    
    \item \textbf{Maximum registered users:} Platform capacity limit (50,000 users)
    
    \item \textbf{Traffic Conditions:} Time-dependent edge weights in road graph representing congestion
    
    \item \textbf{Charging infrastructure:} Number and location of stations with charging rate (5 stations, 7.2 kW)
\end{itemize}

\begin{keybox}
Each of these parameters can be varied in simulation experiments to assess their impact on overall service performance through \textbf{scenario analysis}.
\end{keybox}


\subsection{Performance Indicators}


To evaluate the functioning of the car-sharing system, a set of key performance indicators (KPIs) is defined:

\begin{table}[h]
\centering
\begin{tabular}{@{}lp{9cm}c@{}}
\toprule
\textcolor{keycolor}{\textbf{KPI}} & \textbf{Description} & \textbf{Target} \\
\midrule
Availability Rate & Fraction of requests fulfilled by nearby vehicles & $>$ 80\% \\[0.2cm]
Average Waiting Time & Time between arrival and car acquisition & $<$ 15 min \\[0.2cm]
Vehicle Utilization & Time vehicles spend in active use & 60-80\% \\[0.2cm]
Charging Occupancy & Proportion of time charging stations occupied & $<$ 80\% \\[0.2cm]
Relocation Efficiency & Service improvement vs. operational cost & Maximized \\
\bottomrule
\end{tabular}
\caption{Key Performance Indicators with target values}
\end{table}

Together, these indicators provide a comprehensive picture of both \textcolor{keycolor}{\textbf{user satisfaction}} and \textcolor{keycolor}{\textbf{operational performance}}.

\section{Entities}

In this section, we examine the main entities of the car-sharing system and their interactions.

\begin{itemize}[leftmargin=*, label=\textcolor{primarycolor}{$\bullet$}]

    \item \textbf{User}: contains information such as current location, desired destination, reservation status, and time of arrival.

    \item \textbf{Vehicle}: stores attributes including current position, battery level, state (available, in use, charging, or reserved), and time since last relocation.

    \item \textbf{Charging Station}: characterized by its geographic coordinates, number of charging points, queue of waiting vehicles, and current utilization level.

\end{itemize}


\subsection{Core Data Structures}


To manage these entities and their interactions efficiently, several supporting data structures are employed:


\begin{itemize}

    \item A \textbf{priority queue} to handle the chronological sequence of events, ensuring that the simulation processes occurrences in correct temporal order.

    \item A \textbf{spatial index} or lookup structure (for example, a grid or k-d tree) to enable rapid identification of the nearest available vehicle for a user request.

    \item A \textbf{vehicle list} or dictionary to maintain global vehicle states and allow efficient updates during trips, charging, or relocation.

    \item A \textbf{charging queue} for each station, storing vehicles waiting to recharge and managing service order.

\end{itemize}


These structures collectively enable an efficient event-driven simulation capable of managing large numbers of vehicles and users while preserving spatial and temporal realism.


\section{Conclusion}


The conceptual design developed in this laboratory defines the main elements necessary to simulate a car-sharing system with electric vehicles. It identifies the key parameters influencing system behavior, the performance indicators used to evaluate efficiency and service quality, and the essential dynamics governing user and vehicle interactions. 


The proposed data structures and assumptions provide a solid foundation for subsequent implementation and experimentation. Future extensions may include refined user behavior models, predictive relocation strategies, and the introduction of time-dependent demand to further approximate real-world conditions.


\end{document}